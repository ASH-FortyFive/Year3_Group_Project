The Smart Fridge is a product with sustainability at its core.
Energy usage is kept low by using minimal power consumption components such as ESP32 and
the RPi and further reduced by only processing data when the door is opened.
The backend is also powered by open-source software, meaning everything could be selfhosted if our servers shut down.
Electronic waste is also reduced by presenting information on the user's device instead of a dedicated screen on the fridge.
Our product will also aim to achieve the UN Sustainable Development Goals by minimizing food waste,
freeing up time and encouraging healthy food habits. 

{\bf Goal 2 Zero Hunger:}
The recipes recommended by the app will help the user improve their nutrition by suggesting better meals.
On top of this, food waste will also be reduced at home, by notifying users of upcoming expiration dates.
On average, a person in Europe will waste 100kg of food every year [3].  

{\bf Goal 3 Good Health and Well-being:}
A core part of maintaining a healthy lifestyle is to have the required amount of nutrition.
Without these important nutrients, the likelihood of CDDs is increased.
As mentioned previously, the insights provided by the app help with nutrition.  

{\bf Goal 8 Decent Work and Economic Growth:}
Meal planning and food preparation can take a large amount of time.
The user could instead use this time on themselves and relax.
When stress free, people perform better at work.  

{\bf Goal 12 Responsible Consumption and Production and Goal 13 Climate Action:}
Goal 12 and 13 are approached in a similar way for this product, 
since reducing waste is an example of responsible consumption and lessens the impact of climate change.
In 2007 1.4 billion hectares of land were used to produce food not consumed [3].
To further reduce energy consumption, the user will be reminded to close the fridge by a buzzer. 