The Smart Fridge is a device which tracks the inventory of the user's fridge by using sensors to detect what is going in and out.
The data collected will update live on an app which the user can access remotely.
This means that they never have to guess what is in their fridge and can be given insights about the content of their fridge,
such as recipe recommendation, nutritional facts or reminders about expiry.
There are currently food inventory apps available, however, to use them, 
the user must input the data such as the type of item manually.
What separates the Smart Fridge from these apps is that the Smart Fridge will use computer vision to automatically track the inventory 
and will not interfere with the user's typical use of the fridge. 

The Smart Fridge aims to empower users to make better decisions about their food. 
Be this by providing users with an up-to-date inventory of their fridge, so they do not overstock perishables, 
or by giving reminders about expirations, helping reduce the amount of food. 
This inventory data can also be with existing API to provide the user with recipe recommendations and nutritional information. 

The Grand Challenge we will tackle is 'Artificial Intelligence and data', the contents of someone's fridge is very valuable data, 
that can be used to aid a user's nutrition and health, reduce environmental impact via reducing waste or making better choices, 
and making the average home smarter by leveraging Internet of Things technology. 

The Grand Challenge of 'Ageing society' will also be worked on, as better nutrition
can let people live longer healthier lives with a better diet.

By January 2023 we will have produced a working proof of concept with complete test documentation.  
